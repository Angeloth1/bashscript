% Options for packages loaded elsewhere
\PassOptionsToPackage{unicode}{hyperref}
\PassOptionsToPackage{hyphens}{url}
%
\documentclass[
]{article}
\usepackage{amsmath,amssymb}
\usepackage{lmodern}
\usepackage{iftex}
\ifPDFTeX
  \usepackage[T1]{fontenc}
  \usepackage[utf8]{inputenc}
  \usepackage{textcomp} % provide euro and other symbols
\else % if luatex or xetex
  \usepackage{unicode-math}
  \defaultfontfeatures{Scale=MatchLowercase}
  \defaultfontfeatures[\rmfamily]{Ligatures=TeX,Scale=1}
\fi
% Use upquote if available, for straight quotes in verbatim environments
\IfFileExists{upquote.sty}{\usepackage{upquote}}{}
\IfFileExists{microtype.sty}{% use microtype if available
  \usepackage[]{microtype}
  \UseMicrotypeSet[protrusion]{basicmath} % disable protrusion for tt fonts
}{}
\makeatletter
\@ifundefined{KOMAClassName}{% if non-KOMA class
  \IfFileExists{parskip.sty}{%
    \usepackage{parskip}
  }{% else
    \setlength{\parindent}{0pt}
    \setlength{\parskip}{6pt plus 2pt minus 1pt}}
}{% if KOMA class
  \KOMAoptions{parskip=half}}
\makeatother
\usepackage{xcolor}
\usepackage{longtable,booktabs,array}
\usepackage{calc} % for calculating minipage widths
% Correct order of tables after \paragraph or \subparagraph
\usepackage{etoolbox}
\makeatletter
\patchcmd\longtable{\par}{\if@noskipsec\mbox{}\fi\par}{}{}
\makeatother
% Allow footnotes in longtable head/foot
\IfFileExists{footnotehyper.sty}{\usepackage{footnotehyper}}{\usepackage{footnote}}
\makesavenoteenv{longtable}
\usepackage{graphicx}
\makeatletter
\def\maxwidth{\ifdim\Gin@nat@width>\linewidth\linewidth\else\Gin@nat@width\fi}
\def\maxheight{\ifdim\Gin@nat@height>\textheight\textheight\else\Gin@nat@height\fi}
\makeatother
% Scale images if necessary, so that they will not overflow the page
% margins by default, and it is still possible to overwrite the defaults
% using explicit options in \includegraphics[width, height, ...]{}
\setkeys{Gin}{width=\maxwidth,height=\maxheight,keepaspectratio}
% Set default figure placement to htbp
\makeatletter
\def\fps@figure{htbp}
\makeatother
\setlength{\emergencystretch}{3em} % prevent overfull lines
\providecommand{\tightlist}{%
  \setlength{\itemsep}{0pt}\setlength{\parskip}{0pt}}
\setcounter{secnumdepth}{-\maxdimen} % remove section numbering
\ifLuaTeX
  \usepackage{selnolig}  % disable illegal ligatures
\fi
\IfFileExists{bookmark.sty}{\usepackage{bookmark}}{\usepackage{hyperref}}
\IfFileExists{xurl.sty}{\usepackage{xurl}}{} % add URL line breaks if available
\urlstyle{same} % disable monospaced font for URLs
\hypersetup{
  hidelinks,
  pdfcreator={LaTeX via pandoc}}

\author{}
\date{}

\begin{document}

\hypertarget{ux3b5ux3c0ux3b9ux3baux3bfux3b9ux3bdux3c9ux3bdux3afux3b1-ux3b1ux3bdux3b8ux3c1ux3ceux3c0ux3bfux3c5-ux3c5ux3c0ux3bfux3bbux3bfux3b3ux3b9ux3c3ux3c4ux3ae}{%
\section{Επικοινωνία
Ανθρώπου-Υπολογιστή}\label{ux3b5ux3c0ux3b9ux3baux3bfux3b9ux3bdux3c9ux3bdux3afux3b1-ux3b1ux3bdux3b8ux3c1ux3ceux3c0ux3bfux3c5-ux3c5ux3c0ux3bfux3bbux3bfux3b3ux3b9ux3c3ux3c4ux3ae}}

Ονοματεπώνυμο: Άγγελος Θώμος

ΑΜ: 2019095

\begin{longtable}[]{@{}
  >{\raggedright\arraybackslash}p{(\columnwidth - 6\tabcolsep) * \real{0.2500}}
  >{\raggedright\arraybackslash}p{(\columnwidth - 6\tabcolsep) * \real{0.2500}}
  >{\raggedright\arraybackslash}p{(\columnwidth - 6\tabcolsep) * \real{0.2500}}
  >{\raggedright\arraybackslash}p{(\columnwidth - 6\tabcolsep) * \real{0.2500}}@{}}
\toprule()
\begin{minipage}[b]{\linewidth}\raggedright
Εβδομάδα
\end{minipage} & \begin{minipage}[b]{\linewidth}\raggedright
\href{https://courses-ionio.github.io/help/deliverables/}{Όλα τα
παραδοτέα βρίσκονται στην ίδια σελίδα της τελικής αναφοράς} με τα
προσωπικά στοιχεία σας (Όνομα, ΑΜ, github profile) και μαζί με αυτόν εδώ
τον πίνακα περιεχομένων
\end{minipage} & \begin{minipage}[b]{\linewidth}\raggedright
Σύνδεσμος στην
\href{https://github.com/courses-ionio/help/discussions/categories/show-and-tell}{εβδομαδιαία
παρουσίαση προόδου στις συζητήσεις}
\end{minipage} & \begin{minipage}[b]{\linewidth}\raggedright
Αυτοαξιολόγηση σύμφωνα με τα κριτήρια της αντίστοιχης άσκησης
\end{minipage} \\
\midrule()
\endhead
1 &
\href{https://github.com/courses-ionio/hci/discussions/1794}{Δημιουργία
ομάδας} + \href{https://courses-ionio.github.io/help/guide/}{Φορκ και
δημιουργία σελίδας τελικής αναφοράς},
\href{https://raw.githubusercontent.com/courses-ionio/hci/master/README.md}{προσθήκη
πίνακα περιεχομένων},
\href{https://courses-ionio.github.io/help/intro/}{συγγραφή της
εισαγωγής}, αποστολή της εισαγωγής
\href{https://github.com/courses-ionio/help/discussions/categories/show-and-tell}{για
σχολιασμό στην συζήτηση} και καταγραφή του συνδέσμου συζήτησης δίπλα
--\textgreater{} &
\href{https://github.com/courses-ionio/help/discussions/822}{Συζήτηση},
\href{https://github.com/Second-Time-is-the-Charm}{Ομάδα} & Επιτυχής
ολοκλήρωση, εντός προθεσμίας \\
2 & Άσκηση γραμμής εντολών (linux install) &
\href{https://github.com/courses-ionio/help/discussions/1092}{Συζήτηση},
\href{https://asciinema.org/a/v6iy1N8PzgTMxa3GR3hApOTe9}{asciinema} &
Επιτυχής ολοκλήρωση, εντός προθεσμίας \\
3 & Συμμετοχικό περιεχόμενο A1 &
\href{https://github.com/courses-ionio/help/discussions/1180}{Συζήτηση},\href{https://stitc-site.netlify.app/gallery/artix-os/}{Προσθήκη1},\href{https://stitc-site.netlify.app/gallery/suse-os/}{Προσθήκη2}
& Επιτυχής ολοκλήρωση, εντός προθεσμίας \\
4 & Άσκηση γραμμής εντολών (warm up cli) &
\href{https://github.com/courses-ionio/help/discussions/1323}{Συζήτηση},\href{https://asciinema.org/a/MF11tTiNWWAXU072wriDwqBfr}{fish},\href{https://asciinema.org/a/4SMIyNPkZqWnTF3CEE5qqtN7M}{wttr}
& Επιτυχής ολοκλήρωση, εντός προθεσμίας \\
5 & Συμμετοχικό περιεχόμενο A2 &
\href{https://github.com/courses-ionio/help/discussions/1474}{Συζήτηση},\href{https://github.com/Second-Time-Is-The-Charm/site/blob/master/_slides/os.md}{slide
update},\href{https://github.com/Second-Time-Is-The-Charm/site/blob/master/_timeline/os-apps.md}{timeline
update} & Επιτυχής ολοκλήρωση, εντός προθεσμίας \\
6 & Κατασκευή του βιβλίου Α &
\href{https://github.com/courses-ionio/help/discussions/1613}{Συζήτηση}
& Επιτυχής ολοκλήρωση \\
7 & Συμμετοχικό περιεχόμενο B1 &
\href{https://github.com/courses-ionio/help/discussions/1693}{Συζήτηση},\href{https://site-reme1o673-angeloth1.vercel.app/case-study/bash/}{cs-study}
& \\
8 & Άσκηση γραμμής εντολών & & \\
9 & Συμμετοχικό περιεχόμενο B2 & & \\
10 & Άσκηση γραμμής εντολών & & \\
11 & Κατασκευή του βιβλίου Β & & \\
12 & Τελική αναφορά* & & \\
\bottomrule()
\end{longtable}

\hypertarget{ux3c0ux3b1ux3c1ux3b1ux3b4ux3bfux3c4ux3b5ux3bf-1-ux3b5ux3b9ux3c3ux3b1ux3b3ux3c9ux3b3ux3b7}{%
\subsection{ΠΑΡΑΔΟΤΕΟ 1 \textbar{}
ΕΙΣΑΓΩΓΗ}\label{ux3c0ux3b1ux3c1ux3b1ux3b4ux3bfux3c4ux3b5ux3bf-1-ux3b5ux3b9ux3c3ux3b1ux3b3ux3c9ux3b3ux3b7}}

Μέσο αυτού του μαθήματος στοχεύω να διευρύνω τις γνώσεις μου σχετικά με
την ιστορία του υπολογιστή και πως έφτασε να είναι αυτό που είναι.
Επίσης η χρήση και η εξοικείωση με το github αλλά και με τα arch είναι
σίγουρα μία από τις πιο σημαντικές προσφορές αυτού του μαθήματος όπου
θέλω να τις κάνω κτήμα μου. (προσθήκη στα μέσα του εξαμήνου καθώς έχω
μια πιο ολοκληρομένη εικόνα για το μάθημα) Στο μάθημα αυτό θέλω να μάθω
περισσότερα για την \textless\textless{} μη επικρατούσα
\textgreater\textgreater{} διεπαφή με τον υπολογιστή, δηλαδή για το
command line, καθώς και τα unix based συστήματα. Ερωτήματα που περιμένω
ως το τέλος του 6μήνου να έχω απαντήσει είναι: - Γιατί δεν επικράτησε η
χρήση του υπολογιστή μέσο terminal; - Τι διεπαφή χρησιμοποιούμε σήμερα
και πώς εδρεόθηκε; - Πώς μπορώ να χρησιμοποιήσω τον υπολογιστή όχι σαν
ψηφιακή απεικόνιση του χαρτιού, όπως έχει πει και
\href{https://en.wikipedia.org/wiki/Ted_Nelson}{Ted Nelson} αλλά ως το
εργαλείο που οραματιζόταν τόσο αύτος όσο και άλλοι προπάτορας του
επιστημονικού μας κλάδου;

Τέλος θέλω να μάθω να δουλεύω συνεργατικά και να δω και διευρύνω τις
ικανότητες μου στην διεκπεραίωση ομαδικού και συνεργατικού έργου.

\hypertarget{ux3c0ux3b1ux3c1ux3b1ux3b4ux3bfux3c4ux3b5ux3bf-2-arch-install}{%
\subsection{ΠΑΡΑΔΟΤΕΟ 2 \textbar{} Arch
install}\label{ux3c0ux3b1ux3c1ux3b1ux3b4ux3bfux3c4ux3b5ux3bf-2-arch-install}}

Έκανα εγκατάσταση των Arch Linux σε ένα παλιό σύστημα, η προετιμασία της
συσκευής εγκατάστασης έγινε με sudo dd bs=4M if=`PATH'.iso of=/dev/sd'X'
conv=fdatasync . Στην συνέχεια έκανα κλασσική εγκατάσταση χωρίς UEFI
καθώς το σύστημα μου δεν το υποστηρίζει.
\href{https://asciinema.org/a/v6iy1N8PzgTMxa3GR3hApOTe9}{\includegraphics{https://asciinema.org/a/v6iy1N8PzgTMxa3GR3hApOTe9.svg}}

\hypertarget{ux3c0ux3b1ux3c1ux3b1ux3b4ux3bfux3c4ux3b5ux3bf-3-a1}{%
\subsection{ΠΑΡΑΔΟΤΕΟ 3 \textbar{}
A1}\label{ux3c0ux3b1ux3c1ux3b1ux3b4ux3bfux3c4ux3b5ux3bf-3-a1}}

Σε αυτό το παραδοτέο συνεισφέραμε με τις δικές μας προσθήκες
περιεχομένου στην ιστοσελίδα του μαθήματος. κάνοντας 2 προσθήκες. Οι
προσθήκες αυτές συμπεριλάμβανουν 2 εικόνες με λεζάντες σε αρχεία που
βρίσκονται στο αποθετήριο της σελίδας \texttt{/site/images/} και
\texttt{/site/\_gallery/}. Οι προσθήκες που έκανα έγω είναι οι ακόλουθες
- \href{https://stitc-site.netlify.app/gallery/artix/}{artix} -
\href{https://stitc-site.netlify.app/gallery/suse/}{suse}

και τα pull req. αντίστοιχα για τον ενταγμό τους στην ομάδα -
\href{https://github.com/Second-Time-Is-The-Charm/_gallery/pull/5\#event-7645248395}{\_gallery
pull request} -
\href{https://github.com/Second-Time-Is-The-Charm/images/pull/5\#issuecomment-1287660118}{images
pull request}

\hypertarget{ux3c0ux3b1ux3c1ux3b1ux3b4ux3bfux3c4ux3b5ux3bf-4-warm-up}{%
\subsection{ΠΑΡΑΔΟΤΕΟ 4 \textbar{}
warm-up}\label{ux3c0ux3b1ux3c1ux3b1ux3b4ux3bfux3c4ux3b5ux3bf-4-warm-up}}

Σε αυτό το παραδοτέο έκανα δύο από τις ασκήσεις warm-up καθώς την
εγκατάσταση των arch linux την έχω ολοκληρώσει από το
\href{https://github.com/courses-ionio/help/discussions/1092}{παραδοτέο
2} , οι δύο ασκήσεις όπου επέλεξα είναι

\begin{enumerate}
\def\labelenumi{\arabic{enumi}.}
\tightlist
\item
  εγκατάσταση και επεξεργασία του fish shell
  \href{https://asciinema.org/a/MF11tTiNWWAXU072wriDwqBfr}{\includegraphics{https://asciinema.org/a/MF11tTiNWWAXU072wriDwqBfr.svg}}
\item
  εξοικείωση με το wttr
  \href{https://asciinema.org/a/4SMIyNPkZqWnTF3CEE5qqtN7M}{\includegraphics{https://asciinema.org/a/4SMIyNPkZqWnTF3CEE5qqtN7M.svg}}
\end{enumerate}

\hypertarget{ux3c0ux3b1ux3c1ux3b1ux3b4ux3bfux3c4ux3adux3bf-5-slides-timeline}{%
\subsection{Παραδοτέο 5 \textbar{} slides \&
timeline}\label{ux3c0ux3b1ux3c1ux3b1ux3b4ux3bfux3c4ux3adux3bf-5-slides-timeline}}

\begin{itemize}
\tightlist
\item
  Σε αύτο το παραδοτέο πρόσθεσα υλικό στο
  \href{https://github.com/Second-Time-Is-The-Charm/site/blob/master/_slides/os.md}{slide}
\item
  Αντίστοιχα πρόσθεσα υλικό στο
  \href{https://github.com/Second-Time-Is-The-Charm/site/blob/master/_timeline/os-apps.md}{timeline}
\end{itemize}

Επέλεξα να προσθέσω υλικό και να μην δημιουργήσω δικό μου, διότι
προσπαθόντας κατακτήσω όσες περισσότερες γνώσεις μπορώ από το μάθημα
αλλά συνάμα να παραδώσω να ένα αξιοπρεπές υλικό. Έκρεινα ότι η
θεματολογία των υπάρχων κάλυπταν πλήρος και μου τις προσθήκες που έκανα
στο A1.

Τα links για τις αλλαγές στο site -
\href{https://stitc-site.netlify.app/slides/os/}{slide} -
\href{https://jazzy-khapse-290902.netlify.app/timeline/os-apps/}{timeline}

Τα pull req -
\href{https://github.com/Second-Time-Is-The-Charm/site/pull/7}{slide} -
\href{https://github.com/Second-Time-Is-The-Charm/site/pull/9}{timeline}

\hypertarget{ux3c0ux3b1ux3c1ux3b1ux3b4ux3bfux3c4ux3adux3bf-6-book-entry}{%
\subsection{Παραδοτέο 6 \textbar{} book
entry}\label{ux3c0ux3b1ux3c1ux3b1ux3b4ux3bfux3c4ux3adux3bf-6-book-entry}}

Αρχικά για αυτό το παραδοτέο χρειάστηκε να εξοικειωθώ με δύο αγνώστες
-για εμένα- ως τώρα τεχνολογίες - {[}x{]} Το εργαλίο
\href{https://pandoc.org/}{pandoc} - {[}x{]} Την γλώσσα
\href{https://www.lua.org/}{lua}

Στην συνέχεια κλήθηκα να κάνω τις ακούλουθες προσθήκες στο βιβλίο αφού
το δημιουργήσω σε .tex αρχείο - {[}x{]} δικό μου
\href{https://github.com/Angeloth1/kallipos/blob/master/contribution.lua}{φίλτρο}
- {[}x{]} μετατροπή σε
\href{https://github.com/Angeloth1/kallipos/blob/master/book/book.pdf}{pdf}
- {[}x{]} μια δική μου
\href{https://github.com/Angeloth1/kallipos/blob/master/contribution/unixLin.md}{προσθήκη}

Σε αυτό το παραδοτέο χρειάστηκε να επεξεργαστώ το make-latex.sh έτσι
ώστε να δημιουργεί το βιβλίο σε tex και σε pdf μορφή αλλά και να καλεί
το φίλτρο μου∙ Βέβαια έγιναν κι άλλες αλλαγές στο make-latex για να
υπάρχει feedback. Τέλος αυτό είναι το contribution.md και η αλλαγή στο
βιβλίο.

\hypertarget{ux3c0ux3b1ux3c1ux3b1ux3b4ux3bfux3c4ux3adux3bf-7-case-study}{%
\subsection{Παραδοτέο 7 \textbar{}
case-study}\label{ux3c0ux3b1ux3c1ux3b1ux3b4ux3bfux3c4ux3adux3bf-7-case-study}}

Για το case study μου αποφάσισα να ασχοληθώ με τα shells και
συγκεκριμένα με το \href{https://www.gnu.org/software/bash/}{bash}.
Αναφέρω μερικές από τις βασικές του διαφοροποιήσεις με τα ως τότε
υπάρχοντα shells καθώς και μερικές από τις δυνατότης που προσφέρουν στο
χρήστη τα shells. Οι πηγές όπου χρησιμοποίησα είναι οι ακόλουθες (και
αξίζουν νομίζω να τις μελετήσετε καθώς σε αυτό το case study δεν μπορώ
παρά να αναφέρω το ρεσουμέ από μερικά σημεία από την καθέ μια). -
\href{http://bashcookbook.com/bashinfo/source/bash-4.0/doc/rose94.pdf}{BASH
1} -
\href{https://books.google.gr/books?hl=el\&lr=\&id=dzBCH3x6fYEC\&oi=fnd\&pg=PT7\&dq=bash+shell\&ots=iXl_zalwFU\&sig=Z-EfIYMoAn-QBjrJOzt_x97YQy8\&redir_esc=y\#v=onepage\&q\&f=false}{BASH
2} - \href{https://en.wikipedia.org/wiki/Bash_(Unix_shell)}{BASH 3} -
\href{https://en.wikipedia.org/wiki/POSIX}{POSIX 1} -
\href{https://itsfoss.com/posix/}{POSIX 2}

Τα αρχεία που χρειάστικε να φτιάξω είναι:
\href{https://github.com/Angeloth1/site/blob/master/_case-study/bash.md}{cs-study}
,
\href{https://github.com/Angeloth1/site/blob/master/_includes/cs-bash.md}{includes}.
Καταλήγοντας λοιπόν στο ακόλουθο αποτέλεσμα : -
\href{https://site-nu-wine.vercel.app/case-study/bash/}{bash shell}

αν και θα ήθελα να αναφέρο περισσότερα για την διεπαφή που προσφέρει
συγκεκριμένα το bash, η αλήθεια είναι ότι τα περισσότερα αν όχι όλα τα
shells έχουν περίπου ίδια διεπαφή

\hypertarget{team}{%
\subsection{TEAM}\label{team}}

\href{https://github.com/Second-Time-is-the-Charm}{Second Time Is The
Charm}

\begin{longtable}[]{@{}
  >{\raggedright\arraybackslash}p{(\columnwidth - 8\tabcolsep) * \real{0.1489}}
  >{\raggedright\arraybackslash}p{(\columnwidth - 8\tabcolsep) * \real{0.1064}}
  >{\raggedright\arraybackslash}p{(\columnwidth - 8\tabcolsep) * \real{0.3617}}
  >{\raggedright\arraybackslash}p{(\columnwidth - 8\tabcolsep) * \real{0.2340}}
  >{\raggedright\arraybackslash}p{(\columnwidth - 8\tabcolsep) * \real{0.1489}}@{}}
\toprule()
\begin{minipage}[b]{\linewidth}\raggedright
Members
\end{minipage} & \begin{minipage}[b]{\linewidth}\raggedright
Roles
\end{minipage} & \begin{minipage}[b]{\linewidth}\raggedright
First \& Last name
\end{minipage} & \begin{minipage}[b]{\linewidth}\raggedright
Id
\end{minipage} & \begin{minipage}[b]{\linewidth}\raggedright
Account link
\end{minipage} \\
\midrule()
\endhead
Voltmaister & Admin & Orestis Artinopoulos & P2019153 &
\href{https://github.com/voltmaister}{Know me!} \\
Markedd & Member & Dimitra Markou & P2019170 &
\href{https://github.com/marked-d}{Know me!} \\
NickLitharis & Member & Nikos Litharis & P2019083 &
\href{https://github.com/NickLitharis}{Know me!} \\
KonstantinosTourtsakis & Member & Konstantinos Tourtsakis & P2019140 &
\href{https://github.com/KonstantinosTourtsakis}{Know me!} \\
odysseasp2019060 & Member & Odysseas Oikonomou & P2019060 &
\href{https://github.com/odysseasp2019060/}{Know me!} \\
artopodama & Member & Giannis Anastasopoulos & inf2021017 &
\href{https://github.com/artopodama/}{Know me!} \\
Angeloth1 & Member & Angelos Thomos & P2019095 &
\href{https://github.com/Angeloth1/}{Know me!} \\
\bottomrule()
\end{longtable}

\end{document}
